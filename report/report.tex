\documentclass[a4paper,11pt]{report}
\usepackage[T1]{fontenc}
\usepackage[utf8]{inputenc}
\usepackage[french]{babel}
\usepackage{lmodern}

\title{Refonte du site web du club d'échec de Montpellier}
\author{Maxime Bertrand, Benjamin Hoquy, Sébastien Lacombe, Jason Thibur}

\begin{document}

\newpage

\renewcommand{\contentsname}{Sommaire}
\renewcommand{\chaptername}{}

\maketitle
\tableofcontents

\chapter*{Remerciements}

\chapter*{Introduction}
De nos jours, un site qui n’évolue pas n’attire pas les visiteurs. et lasse les coutumiers. Afin de garder un attrait un site doit se renouveler de temps en temps. De plus, selon l’interface et les technologies utilisées la gestion de son site peut rapidement devenir complexe et fastidieuse.

C’est dans ce contexte que le club d’échecs montpelliérain Echecs club Montpellier souhaite faire une refonte de son site internet afin de lui donner un aspect plus jeune, ajouter une web-boutique et surtout faciliter la gestion de ses publications.

En effet, le site actuel ne permet qu’une gestion minimale et fastidieuse des articles. Avoir un back-office clair afin de gérer ses publications ainsi que les commentaires peut rendre cette tâche bien plus agréable et intuitive.

C’est autour de cette problématique que le club d’échecs nous a contacté pour développer une refonte de leur site internet plus simple d’utilisation et intégrant une boutique, en utilisant la technologie Symfony2.

Après avoir définis le cadre du projet et les besoins généraux de celui-ci dans l’analyse, nous détaillerons l’ensemble des fonctionnalités, l’architecture développées pour arriver au résultat final dans le rapport technique. Enfin nous reviendrons dans le rapport d’activité sur les méthodes ainsi que les différents outils de travail utilisés pour réaliser ce projet.

\part{Analyse}
\chapter{Analyse du contexte}

\part{Conception}

\part{Rapport technique}


\begin{abstract}
Le Projet ECMCHESS consiste à réaliser un site web vitrine pour un club d’échec montpellierain.
Ce dernier devait pouvoir fournir divers informations sur le club, informer ses membres et permettre l’achat de matériel d’échec via une boutique en ligne
Le site a été développé en utilisant le framework Symfony 2 ainsi que des modules notamment Bootstrap et Sonata.
mots-clés : Échecs, Club d’échec, Symfony 2, boutique en ligne.
\end{abstract}

\begin{abstract}
ECMCHESS project consists in developping a website for a chessclub of Montpellier.
It was aimed to show various inforation about the club, display news and events to the club’s members and allow customer to buy chess’s stuff on an online store.
This web site has been developped using the Symfony 2 framework and bundles for instance Bootstrap and Sonata.

key words : Chess, chess club, symfony 2, e-shop.
\end{abstract}

\end{document}
